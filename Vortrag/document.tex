%% LaTeX Beamer presentation template (requires beamer package)
%% see http://bitbucket.org/rivanvx/beamer/wiki/Home
%% idea contributed by H. Turgut Uyar
%% template based on a template by Till Tantau
%% this template is still evolving - it might differ in future releases!

\documentclass{beamer}

\mode<presentation>
{
\usetheme{Frankfurt}

\setbeamercovered{transparent}
}

\usepackage[english]{babel}
\usepackage[latin1]{inputenc}

% font definitions, try \usepackage{ae} instead of the following
% three lines if you don't like this look
\usepackage{mathptmx}
\usepackage[scaled=.90]{helvet}
\usepackage{courier}


\usepackage[T1]{fontenc}

\usepackage{amsmath,amssymb}



\title{Adaptive Markov Chain Monte Carlo Methods}

%\subtitle{}

% - Use the \inst{?} command only if the authors have different
%   affiliation.
%\author{F.~Author\inst{1} \and S.~Another\inst{2}}
\author{Stefan Junghans\\ Mitja Richter}

% - Use the \inst command only if there are several affiliations.
% - Keep it simple, no one is interested in your street address.
\institute[Universities of]
{
TU Berlin
}

\date{25.07.2014}


% This is only inserted into the PDF information catalog. Can be left
% out.
\subject{Adaptive Markov Chain Monte Carlo Methods}



% If you have a file called "university-logo-filename.xxx", where xxx
% is a graphic format that can be processed by latex or pdflatex,
% resp., then you can add a logo as follows:

% \pgfdeclareimage[height=0.5cm]{university-logo}{university-logo-filename}
% \logo{\pgfuseimage{university-logo}}



% Delete this, if you do not want the table of contents to pop up at
% the beginning of each subsection:
\AtBeginSubsection[]
{
\begin{frame}<beamer>
\frametitle{Outline}
\tableofcontents[currentsection,currentsubsection]
\end{frame}
}

% If you wish to uncover everything in a step-wise fashion, uncomment
% the following command:

%\beamerdefaultoverlayspecification{<+->}

\begin{document}

\begin{frame}
\titlepage
\end{frame}

\begin{frame}
\frametitle{Outline}
\tableofcontents
% You might wish to add the option [pausesections]
\end{frame}


\section{Introduction}




\subsection{Markov Chain Monte Carlo}
\begin{frame}
\frametitle{Markov Chain Monte Carlo}

\end{frame}


\subsection{Why adaptive?}
\begin{frame}
\frametitle{Why adaptive?}

\end{frame}


\subsection{Criteria}
\begin{frame}
\frametitle{Criteria}
\begin{block}{acceptance rate}
\begin{itemize}
  \item $\alpha = \frac{\# accepted}{\# samples}$
  \item $0.44$ for univariate distributions
  \item $0.234$ for $d\geq 5$ dimensions
\end{itemize}

\end{block}
\begin{block}{suboptimality}
\begin{itemize}
  \item $b = d\cdot \frac{\sum_{i=1}^d \lambda_i^{-2}}{(\sum_{i=1}^d
  \lambda_i^{-1})^2}$
  \item $\lambda_i$ are eigenvalues of $\Sigma_p^{0.5}\Sigma^{-0.5}$
  \item $\Sigma_p$ : proposal covariance matrix
  \item $\Sigma_p$ : target covariance matrix
  \item usually $b>1$, optimum at $b=1$
\end{itemize}

\end{block}
\end{frame}

\begin{frame}
\frametitle{Criteria}
\begin{block}{autocorrelation time}
\begin{itemize}
  \item $ACT = 1 + 2\sum_{i=1}^{\infty} autocorr(i)$
  \item $autocorr(k) = 1/((n-k)\cdot v)\cdot 
  \sum_{t=1}^{n-k}(x^t-\mu)(x^{t+k}-\mu)$
  \item the lower the better!
\end{itemize}

\end{block}
\begin{block}{average squared jump distance}
\begin{itemize}
  \item $ASJD = \sum_{k=1}^\infty \sqrt{\sum_{i=0}^d (x_{(i)}^k-x_{(i)}^{k+1})^2}$
  \item the higher the better!
\end{itemize}

\end{block}
\end{frame}




\section{Methods}


\subsection{Metropolis Hastings}


\begin{frame}
\frametitle{Metropolis Hastings}

\begin{block}{Why do we need it?}
\begin{itemize}
\item target distribution $f(x)$ that we cannot directly sample from
\item $X \sim c \cdot f(x)$, where $c$ is unknown
\end{itemize}
\end{block}

\end{frame}


\begin{frame}
\frametitle{Metropolis Hastings}

\begin{block}{Why do we need it?}
\begin{itemize}
\item target distribution $f(x)$ that we cannot directly sample from
\item $f(x)$ proportional to $c$, which is unknown
\end{itemize}
\end{block}

\begin{block}{Ingredients}
\begin{itemize}
\item $f(x)$ : target density
\item $x_0$ : start point (random)
\item $q(x|y)$ : proposal density, often $N(\mu, \Sigma)$ \\
\item[] \quad \quad  \quad  \quad  ($\mu = y \Rightarrow$ ``Random Walk
Metropolis Hastings'')
\end{itemize}
\end{block}

\end{frame}

\begin{frame}
\frametitle{Metropolis Hastings}

\begin{block}{Steps}
\begin{enumerate}
  \item current sample : $x'$
  \item[]
\item sample $x^* \sim q(x | x')$
\item calculate acceptance $\alpha$:
\[\alpha = min\left(1, \frac{f(x^*)}{f(x')} \cdot
\frac{q(x'|x^*)}{q(x^*|x')} \right)\]

\item accept proposal $x^*$ with probability $\alpha$
\item if accepted : $x' = x^*$
\item repeat
\end{enumerate}
\end{block}
\end{frame}


\begin{frame}
\frametitle{Metropolis Hastings}
\begin{block}{Example}
$N(\mu, \Sigma), \quad \mu = \begin{pmatrix} -4 \\ -2
\end{pmatrix} , \quad \Sigma = \begin{pmatrix} 6,-1 \\ -1,2
\end{pmatrix}$
\end{block}
\end{frame}
 
\subsection{Adaptive Metropolis}
\begin{frame}
\frametitle{Adaptive Metropolis}
\begin{block}{Why do we need it?}
\begin{itemize}
\item If Metropolis Hastings has bad settings, the acceptance rate will be
inferior!
\item high-dimensional target distributions
\item correlated dimensions with different variances
\end{itemize}
\end{block}

\end{frame}


\begin{frame}
\frametitle{Adaptive Metropolis}

\begin{block}{Ingredients}
\begin{itemize}
\item $f(x)$ : target density
\item $x_0$ : start point (random)
\item $q(x|y)\sim N(y, 0.1\cdot I_d / d)$ : proposal density
\item[]
\item $q'(x|y) \sim N(y, 2.38^2\cdot \Sigma_n /
d)$ : ``adapting" proposal density
\begin{itemize}
  \item $2.38^2$ delivers the best results in certain environments
  \item $\Sigma_n$ is the estimated covariance matrix of the current samples 
\end{itemize}
\item $\beta$ : adaption parameter, e.g. $\beta = 0.05$ 
\end{itemize}
\end{block}

\end{frame}

 


\begin{frame}
\frametitle{Adaptive Metropolis}

\begin{block}{Steps}
\begin{enumerate}
  \item current sample : $x'$
  \item[]
\item if $n\leq 2d$ : sample $x^* \sim q(x | x')$
\item if $n> 2d$ : sample $x^* \sim (1-\beta) \cdot q'(x | x') +
\beta \cdot q(x | x')$
\item calculate acceptance $\alpha$:
\[\alpha = min\left(1, \frac{f(x^*)}{f(x')} \right)\]

\item accept proposal $x^*$ with probability $\alpha$
\item if accepted: $x' = x^*$
\item repeat
\end{enumerate}
\end{block}
\end{frame}


\begin{frame}
\frametitle{Adaptive Metropolis}
\begin{block}{Example}
$N(\mu, \Sigma), \quad \mu = \begin{pmatrix} -4 \\ -2
\end{pmatrix} , \quad \Sigma = \begin{pmatrix} 6,-1 \\ -1,2
\end{pmatrix}$
\end{block}
\end{frame}

\subsection{Adaptive Gibbs (Metropolis within Gibbs)}
\begin{frame}

\end{frame} 

\subsection{Regional Metropolis Hastings}
\begin{frame}

\end{frame} 
\subsection{Results}

\begin{frame}
\frametitle{Model}
\end{frame}

\section{Summary}
\subsection{Summary}
\begin{frame}
\frametitle{Summary}

\begin{block}{Review}
\begin{itemize}
\item automated tuning algorithms yield promising results
\item well picked proposal functions are crucial for success
\item in high dimensional fields to difficult to do it by hand
\end{itemize}
\end{block}

\begin{block}{Problems}
\begin{itemize}
\item how robust are these algorithms
\item how determine other parameters such as $\delta(n)$
\item how to determine these for individual problems 
\item what method suits the problem the most
\end{itemize}
\end{block}


\end{frame}
 
\section{References}
\nocite{*}
\begin{frame}[allowframebreaks]
        \frametitle{References}
        \bibliographystyle{amsalpha}
        \bibliography{AMCM} 
\end{frame}

\end{document}
