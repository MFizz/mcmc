
\subsection{Regional Metropolis Hastings}
\begin{frame}
\frametitle{Regional Metropolis Hastings}
\begin{block}{Why do we need it?}
\begin{itemize}
\item target distribution is multimodal with different relative mass
\item no "universal" good proposal
\item need to adapt to different regions of the state space
\end{itemize}
\end{block}
\end{frame}

\begin{frame}
\begin{block}{Ingredients}
\begin{itemize}
\item $f(x)$ : target density\\
consisting of different high probability regions
\item $\chi = \chi_1 \stackrel{\cdot}{\cup} \chi_2 \stackrel{\cdot}{\cup}$ ... disjoint partitions of the state space 
\item $Q_i(x,\cdot) = N(x, exp(2a_i))$: individual proposals for every region\\
$x \in \chi_i \rightarrow \sigma^2_x = e^{2a_i}$ and $y \in \chi_j \rightarrow \sigma^2_y e^{2a_j}$
\item $x_0$ : start point (random)
\item $\alpha_i$ : individual acceptance rate for every region
\item $a_i$ : individual learning parameters for every region
\item $\delta(n)$ adaption method for these parameters

\end{itemize}
\end{block}
\end{frame}

\begin{frame}
\begin{block}{Segmentation}
\begin{itemize}
\item for Segmentation either use
\begin{itemize}
\item $S_k = {x: argmax N(x;\mu_k,\Sigma_k) = k}$ 
or
\item predefined segments by experimentation\\
\end{itemize}
\item seldom enough knowledge to perfectly separate\\

\item to compensate the proposal integrates all $Q_i$
\end{itemize}
\end{block}

\end{frame}

\begin{frame}
\begin{block}{Proposals}
\begin{itemize}
\item the resulting proposal\\

$Q_i(x,\cdot) = \Sigma_{i=1}^n 1_{\chi_i}(x)[\Sigma_{j=1}^n\lambda_j^{(i)}N(x, exp(2a_j))] $\\
where $\Sigma \lambda_j^{(i)} = 1 $
\item adaption parameter are chosen to reflect which proposal is more appropriate in this region

\item proposal for $\lambda_j^{(i)} = \frac{k_j^{(i)}(t)}{\Sigma_{h=1}^{(n)}k_h^{(i)}(t)}$\\ 
where $k_j^{(i)}$ is \# accepted states up to time $t$
\item this would favor small steps so we integrate the average squared jump distance $d_j^{(i)}$
\[
   \lambda_j^{(i)}= 
\begin{cases}
    \frac{d_j^{(i)}(t)}{\Sigma_{h=1}^{(n)}d_h^{(i)}(t)},& if \Sigma_{h=1}^{(n)}d_h^{(i)}(t) > 0\\
    1/2,              & otherwise
\end{cases}
\]

\end{itemize}
\end{block}

\end{frame}

\begin{frame}
\begin{block}{acceptance rate}
\begin{itemize}
\item individual acceptance rates for every region depending on region-traversing:
for example with to regions:
\[
   \alpha(x,y)=min 1, 
\begin{cases}
    \frac{f(y)}{f(x)},& if x,y \in \chi_i\\
    \frac{f(y)Q_1(x|y)}{f(x)Q_2(y|x)},  & if x \in S_2,y \in S_1\\
    \frac{f(y)Q_2(x|y)}{f(x)Q_1(y|x)},  & if x \in S_1,y \in S_2
\end{cases}
\]
\item or more specific:\\
$\alpha(x,y)  = min[1,\frac{f(y)}{f(x)} exp(d(a_x-a_y)-\frac{1}{2}(x-y)^2[exp(-2a_j)-exp(-2a_i)])]$
where $d$ denotes the dimensionality of f(x)


\end{itemize}
\end{block}

\end{frame}

\begin{frame}
\begin{block}{delta function}
\begin{itemize} 
\item the theoretic ideal acceptance rate in higher dimensions is 0.234
\item to achieve ideal acceptance rate we adapt $a_i$ for every region after 100 samples
\item proposal for $/delta(n)$ where $n$ is the $n^{th}$ batch of 100 samples:\\
$\delta(n) = min(0.01,n^{-1/2}$
\item if after 100 samples real acceptance rate $>$ ideal the relevant parameter is increased by $\delta(n)$
\item if acceptance rate $<$ ideal it is decreased
\item if there where no proposals in for this region in the entire badge it will not be altered 
\end{itemize}
\end{block}
\end{frame}